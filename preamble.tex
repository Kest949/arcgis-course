\usepackage{amsthm}
\usepackage{fontspec}
\setmonofont{Menlo}
\usepackage[english,main=russian]{babel}
\usepackage{fancyhdr}
\pagestyle{fancy}

\usepackage[fit]{truncate}
\fancyhead[LE,RO]{\nouppercase{\truncate{0.5\headwidth}{\rightmark}}}
\fancyhead[LO,RE]{\nouppercase{\truncate{0.5\headwidth}{\leftmark}}}

\usepackage{setspace}
\onehalfspacing

\setlength\parindent{20pt}
\setlength\parskip{0em}

\makeatletter
\def\thm@space@setup{%
  \thm@preskip=8pt plus 2pt minus 4pt
  \thm@postskip=\thm@preskip
}

\DeclareSymbolFont{cyrletters}{\encodingdefault}{\familydefault}{m}{it}
\newcommand{\makecyrmathletter}[1]{%
  \begingroup\lccode`a=#1\lowercase{\endgroup
  \Umathcode`a}="0 \csname symcyrletters\endcsname\space #1
}
\count255="409
\loop\ifnum\count255<"44F
  \advance\count255 by 1
  \makecyrmathletter{\count255}
\repeat

\DeclareSourcemap{
  \maps[datatype=bibtex]{
    \map{
      \step[fieldsource=langid, match=russian, final]
      \step[fieldset=presort, fieldvalue={a}]
    }
    \map{
      \step[fieldsource=langid, notmatch=russian, final]
      \step[fieldset=presort, fieldvalue={z}]
    }
  }
}

\renewcommand{\partname}{Часть}
\renewcommand{\contentsname}{Содержание}
\renewcommand{\chaptername}{Упражнение}
\renewcommand{\figurename}{Рис.}
\makeatother